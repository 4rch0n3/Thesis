\documentclass{article}
\usepackage{graphicx} % Required for inserting images
\usepackage[greek,english]{babel}
\usepackage{alphabeta}
\usepackage[utf8]{inputenc}
\usepackage[LGR,T1]{fontenc}
\usepackage{hyperref}

\title{Thesis Draft}
\author{nikiforidisyiannis }
\date{2025}

\begin{document}

\maketitle

\tableofcontents
\clearpage

\section{Εισαγωγή}

\section{Εξέλιξη τεχνολογιών}

\subsection{Οι ψηφιακοί βοηθοί ως πληροφοριακά εργαλεία}

\subsubsection{Σύντομη ιστορική αναδρομή και ορισμός}
\par Ως chatbot μπορούμε να ορίσουμε ένα σύστημα που επιτρέπει την αλληλεπίδραση ανθρώπου-υπολογιστή, μέσω φυσικής γλώσσας (natural language). Το 1950 ο Alan Turing πρότεινε το Turing Test θέτοντας το ερώτημα για το αν οι μηχανές μπορούν να σκεφτούν και από τότε η ιδέα ενός συνομιλιακού συστήματος (ή chatbot) άρχισε να αποκτά ευρύτερη αναγνώριση \cite{turing-1950}. Το πρώτο chatbot αναπτύχθηκε το 1966 με το όνομα Eliza και μέσω μεθόδων ταιριάσματος προτύπων (pattern matching) και με μηχανισμό απόκρισης βασισμένο σε πρότυπα (template-based response mechanism) παρείχε απαντήσεις στους χρήστες ως ψυχοθεραπευτής επιστρέφοντας τις εισόδους τους σε μορφή ερώτησης, προσομοιώνοντας ανθρώπινο λόγο \cite{weizenbaum-1966}. Το 1972 αναπτύχθηκε το PARRY που μέσω μιας πρώτης μορφής ανάλυσης συναισθήματος (sentiment analysis) προσπαθούσε να καταλάβει την πρόθεση του χρήστη και απαντούσε ανάλογα \cite{colby-1981}. Το ALICE, που αναπτύχθηκε το 1995, κατάφερε να κερδίσει το βραβείο Loebner τις χρονιές 2000, 2001 και 2004 ως ο “πιο ανθρώπινος υπολογιστής”. Έκανε χρήση μεθόδων ταιριάσματος προτύπων και εποπτευόμενης μάθησης (supervised learning) με τους προγραμματιστές να έχουν πρόσβαση στη δομή του chatbot μέσω της γλώσσας σήμανσης AIML (Artificial Intelligence Markup Language) \cite{wallace-2007}.
\par Τα σύγχρονα chatbot, όπως τα Apple Siri, Google Gemini, Microsoft Copilot και OpenAI ChatGPT, έχουν ρόλο ψηφιακών βοηθών στοχεύοντας όχι απλώς στην προσομοίωση της ανθρώπινης επικοινωνίας με τον χρήστη αλλά στην αυτοματοποίηση απλών καθημερινών καθηκόντων του και στην άμεση, και όσο το δυνατόν πιο ακριβή, παροχή των πληροφοριών που αναζητεί.

\subsubsection{Βασικές έννοιες}
\par Το ταίριασμα προτύπων αποτελεί μια από τις πιο θεμελιώδεις τεχνικές στην ανάπτυξη των chatbot. Βασίζεται στην αντιστοίχιση της εισερχόμενης φυσικής γλώσσας του χρήστη με προκαθορισμένα γλωσσικά μοτίβα (patterns) ώστε να επιλεγεί η κατάλληλη απάντηση από το σύστημα. Η τεχνική αυτή οργανώνεται συνήθως σε μορφή ζευγών “ερέθισμα-αντίδραση” (stimulus-response), όπου κάθε είσοδος που ταιριάζει με ένα συγκεκριμένο μοτίβο ενεργοποιεί ένα αντίστοιχο πρότυπο απάντησης \cite{ramesh-2017}.
\par Η Γλώσσα Σήμανσης Τεχνητής Νοημοσύνης (AIML) είναι μια επεκτάσιμη γλώσσα σήμανσης που αναπτύχθηκε με σκοπό την κατασκευή συστημάτων που βασίζονται σε κανόνες προτύπων. Χρησιμοποιεί κατηγορίες (category) καθεμία από τις οποίες περιλαμβάνει ένα μοτίβο εισόδου (pattern) και ένα αντίστοιχο πρότυπο απόκρισης (template). Το σύστημα αυτό επιτρέπει την κατασκευή απλών αλλά λειτουργικών διαλόγων χωρίς τη χρήση αλγορίθμων μηχανικής μάθησης. Είναι χρήσιμη σε περιβάλλοντα με σαφώς καθορισμένο πεδίο εφαρμογής, καθώς οι απαντήσεις είναι προβλέψιμες και επιτρέπει άμεση παρέμβαση του προγραμματιστή στη διαχείριση του διαλόγου \cite{marietto-2013}.
\par Η Επεξεργασία Φυσικής Γλώσσας (Natural Language Processing, NLP) εστιάζει στη μελέτη και στην ανάπτυξη υπολογιστικών μεθόδων για την κατανόηση, επεξεργασία και παραγωγή φυσικής γλώσσας. Τα περισσότερα σύγχρονα συστήματα NLP βασίζονται σε τεχνικές μηχανικής μάθησης και βαθιάς μάθησης, αξιοποιώντας μεγάλα σύνολα δεδομένων για να μάθουν πρότυπα της ζητούμενης γλώσσας. Είναι απαραίτητη για την ανάλυση των εισερχόμενων μηνυμάτων του χρήστη ώστε να εντοπιστεί η σημασία και να παραχθεί η κατάλληλη απόκριση με φυσιολογική ροή και γραμματική και λογική συνοχή \cite{adamopoulou-2020}.
\par Η Κατανόηση Φυσικής Γλώσσας (Natural Language Understanding, NLU) αποτελεί υποκλάδο της NLP και επικεντρώνεται στην κατανόηση της πρόθεσης του χρήστη και στην εξαγωγή σημασιολογικής πληροφορίας από την φυσική γλώσσα. Στο πλαίσιο των chatbot είναι υπεύθυνη για την αναγνώριση της πρόθεσης (intent) του χρήστη και για την εξαγωγή οντοτήτων (entities), δηλαδή λέξεων ή φράσεων που περιέχουν κρίσιμη πληροφορία για την εκτέλεση της επιθυμητής ενέργειας. Η αποτελεσματικότητα της καθορίζει σε μεγάλο βαθμό την ποιότητα και την επιτυχία του διαλόγου, αφού επιτρέπει στο σύστημα να ερμηνεύσει ασαφείς, μη τυποποιημένες ή και ατελείς εισόδους του χρήστη με ακρίβεια \cite{adamopoulou-2020}.

\subsubsection{Κατηγοριοποίηση και σχεδιασμός}
\par Ανάλογα με το γνωστικό πεδίο (knowledge domain), δηλαδή την πληροφορία που έχει στη διάθεσή του ή τα δεδομένα πάνω στα οποία έχει εκπαιδευτεί, ένα chatbot μπορεί να είναι ανοιχτού πεδίου (open domain), που μπορεί να απαντήσει σε ερωτήσεις γενικού περιεχομένου ή κλειστού πεδίου (closed domain), που είναι εξειδικευμένο πάνω σε ένα μόνο θέμα, προσφέροντας πιο ακριβείς απαντήσεις για αυτό θυσιάζοντας όμως την ποιότητα των απαντήσεων σε γενικά θέματα \cite{adamopoulou-2020}.
\par Ανάλογα με τον τρόπο που επεξεργάζονται τις εισόδους και την παραγωγή των απαντήσεων, μπορούν να χωριστούν σε μοντέλα βασισμένα σε κανόνες (rule-based) και παραγωγικά μοντέλα (generative). Τα rule-based μπορούν να χρησιμοποιηθούν κυρίως ως πληροφοριακά εργαλεία αφού μπορούν να παρέχουν προκαθορισμένες απαντήσεις στις ερωτήσεις του χρήστη μέσω pattern matching αλλά υστερούν σε περιπτώσεις που η είσοδος του χρήστη δεν είναι ξεκάθαρη ή βρίσκεται εκτός του γνωστικού τους πεδίου ή απαιτεί πρόσβαση σε προηγούμενες εισόδους του χρήστη, μιας και τα περισσότερα δεν έχουν μνήμη \cite{adamopoulou-2020}. Τα generative μοντέλα παράγονται με τη χρήση αλγορίθμων μηχανικής μάθησης και τεχνικών βαθιάς μάθησης, με αποτέλεσμα να φαίνονται πιο ανθρώπινα, να μπορούν να ερμηνεύσουν το ζητούμενο του χρήστη καλύτερα χρησιμοποιώντας και παλαιότερες εισόδους, ακόμα και αν υπάρχει κάποια ασάφεια ή αν είναι μη-τυποποιημένη \cite{adamopoulou-2020}. Ωστόσο είναι πιο περίπλοκα στον σχεδιασμό τους, τόσο λόγω των συνόλων δεδομένων που χρειάζονται για την εκπαίδευσή τους, των βελτιστοποιήσεων (tuning) που θα χρειαστούν, αλλά και του μεγαλύτερου υπολογιστικού κόστους.
\par Σε ορισμένες περιπτώσεις είναι σκόπιμη η συμμετοχή ενός ανθρώπινου παράγοντα σε κάποιο σημείο της λειτουργίας του chatbot (human-aided). Ο ρόλος των επιβλεπόντων είναι να επεμβαίνουν σε περιπτώσεις που το chatbot χρειάζεται περαιτέρω διευκρινίσεις ή βοήθεια λόγω κάποιας έλλειψης στο γνωστικό του πεδίο ή περιορισμού της αρχιτεκτονικής του \cite{kucherbaev-2018}. Τα human-aided chatbots έχουν ιδιαίτερα πλεονεκτήματα σε περιπτώσεις όπου η ακρίβεια της απάντησης έχει μεγάλη σημασία, όπως σε ερωτήσεις για νομικά θέματα, και δεν αρκούν οι γνώσεις ενός απλού χρήστη για να κατευθύνουν σωστά το μοντέλο ώστε να λάβουν την καλύτερη δυνατή απάντηση στο ζητούμενο τους. Σε αντίθεση με τα πλήρως αυτόνομα chatbot, ο ανθρώπινος παράγοντας σημαίνει ότι το κόστος λειτουργίας τους είναι υψηλότερο και η κλιμάκωση τους δυσκολότερη.
\par Ακόμα μπορούν να κατηγοριοποιηθούν ανάλογα με τα δικαιώματα που παρέχει η πλατφόρμα ανάπτυξής τους. Tα μοντέλα ανοικτού κώδικα (open-source) παρέχουν στον προγραμματιστή τη δυνατότητα να επεμβαίνει στις περισσότερες πτυχές της λειτουργίας του chatbot και να συνεισφέρει και στην ανάπτυξη του ίδιου του μοντέλου. Από την άλλη, τα chatbot κλειστού κώδικα (closed-source) λειτουργούν ώς μαύρα κουτιά (black boxes) χωρίς να είναι δυνατή η πλήρη πρόσβαση ή η παρέμβαση στον βασικό πηγαίο κώδικα. Αυτό μπορεί να αποτελέσει σοβαρό μειονέκτημα σε εφαρμογές με αυξημένες απαιτήσεις εξατομίκευσης, απορρήτου ή διαφάνειας.
\begin{figure}[h]
    \centering
    \includegraphics[width=1\linewidth]{figures/2.1.cbt_arch.png}
    \caption{Βασική αρχιτεκτονική ενός chatbot.}
    \label{fig:enter-label}
\end{figure}

\subsubsection{Σημασία για τον νομικό τομέα}
\par Στον Νομικό τομέα, η πληθώρα του διαθέσιμου υλικού βρίσκεται σε μορφή κειμένου, όπως έγγραφα δικαστικών αποφάσεων, συμβάσεις και νομικές γνωμοδοτήσεις. Η Νομική Τεχνητή Νοημοσύνη (Legal Artificial Intelligence, LegalAI) μπορεί να μειώσει τον χρόνο και την προσπάθεια που καταβάλλουν οι επαγγελματίες του τομέα για την κατανόηση και επεξεργασία νομικών κειμένων, αλλά και να παρέχει μια αξιόπιστη πρώτη γνώμη σε όσους δεν είναι εξοικειωμένοι με τον νομικό τομέα, λειτουργώντας ως μια προσιτή μορφή νομικής βοήθειας.
\par Το Legal Judgment Prediction (LJP), είναι μία από τις σημαντικότερες προκλήσεις στον τομέα, ιδίως στο πλαίσιο του Αστικού Δικαίου. Στο σύστημα αυτό, η έκβαση μιας δικαστικής απόφασης καθορίζεται με βάση τα γεγονότα της υπόθεσης και τις αντίστοιχες διατάξεις της ισχύουσας νομοθεσίας. Έτσι, κυρώσεις επιβάλλονται μόνο σε περίπτωση ύπαρξης ενεργειών που ρητά απαγορεύονται από τον νόμο. Το έργο της LJP λοιπόν επικεντρώνεται στην επεξεργασία των γεγονότων και των σχετικών άρθρων της νομοθεσίας που σκοπεύει στην πρόβλεψη δικαστικών αποφάσεων.
\par Στις χώρες που ακολουθούν το σύστημα του Κοινού Δικαίου, οι δικαστικές αποφάσεις βασίζονται σε προηγούμενες παρόμοιες υποθέσεις οι οποίες λειτουργούν ως δεσμευτικά νομικά προηγούμενα. Κατά συνέπεια, ο εντοπισμός της πλέον συναφούς προηγούμενης υπόθεσης αποτελεί θεμελιώδες ζήτημα για την απόδοση δικαιοσύνης. Για την ακρίβεια της πρόβλεψης των αποφάσεων σε αυτό το σύστημα, το Similar Case Matching (SCM), έχει αναδειχθεί σε κομβικό ερευνητικό αντικείμενο στο πεδίο της LegalAI. Η SCM εστιάζει στον εντοπισμό ζευγών υποθέσεων με ουσιαστική ομοιότητα, η οποία μπορεί να οριστεί με διάφορους τρόπους, ανάλογα με το νομικό και πραγματολογικό πλαίσιο. Η SCM εστιάζει στον εντοπισμό ζευγών υποθέσεων με ουσιαστική ομοιότητα, η οποία μπορεί να οριστεί με διάφορους τρόπους, ανάλογα με το νομικό και πραγματολογικό πλαίσιο. Η διαδικασία αυτή απαιτεί την κατάλληλη μοντελοποίηση των συσχετίσεων μεταξύ υποθέσεων αξιοποιώντας πληροφορίες από τα γεγονότα, τα νομικά στοιχεία και την επιχειρηματολογία.
\par Το Legal Question Answering (LQA) δημιουργήθηκε για να προσφέρει αυτόματα απαντήσεις στα ερωτήματα των χρηστών. Η παροχή αξιόπιστων και ποιοτικών νομικών συμβουλών προς μη ειδικούς αποτελεί ένα από τα σημαντικότερα καθήκοντα των επαγγελματιών του νομικού κλάδου. Ωστόσο, η περιορισμένη διαθεσιμότητα νομικών επαγγελματιών καθιστά συχνά δύσκολη την ισότιμη και καθολική πρόσβαση σε εξειδικευμένες νομικές υπηρεσίες. Στο πλαίσιο του LQA, τα ερωτήματα εμφανίζουν σημαντική ποικιλομορφία. Ορισμένα επικεντρώνονται στην επεξήγηση νομικών εννοιών, ενώ άλλα απαιτούν την κατανόηση και ανάλυση συγκεκριμένων δικαστικών υποθέσεων. Επιπλέον, η διατύπωση των ερωτήσεων διαφέρει αισθητά μεταξύ νομικών επαγγελματιών και μη ειδικών, ιδίως όσον αφορά τη χρήση νομικής ορολογίας, οπότε ένα σύστημα το οποίο μπορεί να ανταποκρυθεί σε αυτές τις προκλήσεις αποτελεί μια απλή, αλλά σημαντική εφαρμογή της Νομικής Τεχνιτής Νοημοσύνης \cite{zhong-2020}.

\subsubsection{Συστήματα ερωτήσεων - απαντήσεων}
\cite{abdallah-2023}

\subsubsection{Εφαρμογές σε σύγχρονα μοντέλα επεξεργασίας φυσικής γλώσσας}
\parΤο ChatGPT είναι ένα μεγάλο γλωσσικό μοντέλο (Large Language Model, LLM), πάνω στο οποίο έγινε μια μελέτη για την επίδοσή του στις τελικές εξετάσεις της Νομικής Σχολής του Πανεπιστημίου της Μινεσότα \cite{choi-2023}. Εξετάστηκαν τέσσερα μαθήματα, κλιμακούμενης δυσκολίας, με ερωτήσεις κλειστού τύπου ή με ερωτήσεις ανοιχτού τύπου ή και τα δύο. Με τη χρήση του ChatGPT παράχθηκαν απαντήσεις στα ζητούμενα θέματα και αναμείχθηκαν με τις απαντήσεις που δόθηκαν από τους φοιτητές. Για λόγους σύγκρισης, το διάστημα που διεξήχθει η έρευνα, η Νομική Σχολή του Πανεπιστημίου της Μινεσότα βρισκόταν στη 16η θέση μεταξύ των Νομικών Σχολών της Αμερικής. Μετά το πέρας των εξετάσεων, το chatbot κατάφερε να περάσει όλα τα μαθήματα με συνολικό βαθμό C+. Είχε καλύτερη απόδοση στις ερωτήσεις ανοιχτού τύπου, ενώ στις ερωτήσεις κλειστού τύπου, παρότι ήταν χειρότερη, παρέμενε καλύτερη από την τυχαία επιλογή απαντήσεων. Ωστόσο στις ερωτήσεις κλειστού τύπου που περιείχαν μαθηματικές πράξεις, το ποσοστό των σωστών απαντήσεων ήταν το ίδιο με το αν είχαν απαντηθεί στην τύχη. Κατά τη συγγραφή δοκιμίων, επέδειξε ισχυρή κατανόηση των βασικών νομικών κανόνων και είχε σταθερή οργάνωση και σύνθεση, ωστόσο σε λιγότερο δομημένες ερωτήσεις παρουσιάζει δυσκολίες στο να εντοπίσει τα ζητούμενα, σε πολύ μεγαλύτερο βαθμό από τον μέσο φοιτητή. Με την καλύτερη σχεδίαση των ερωτημάτων (prompt engineering) που τέθηκαν στο chatbot, όπως περιορισμό στην έκταση της απάντησης, εξακρίβωση εννοιών,  ρητές οδηγίες για χρήση πραγματικών δικαστικών υποθέσεων, και επισήμανση του “ακαδημαϊκού” ύφους που θα πρέπει να έχει η απάντησή του, αναμένεται πως θα είχε μεγαλύτερη επιτυχία \cite{choi-2023}. Παρά τις ασυνέπειες στην απόδοση του ChatGPT, γλωσσικά μοντέλα τέτοια κατηγορίας μπορούν να αποτελέσουν σημαντικά εργαλεία για ασκούμενους δικηγόρους στο μέλλον, καθώς και για ανθρώπους που δεν ανήκουν στο νομικό τομέα αλλά επιδιώκουν ευρύτερη κατανόηση των σχετικών νομικών θεματολογιών.
\par Συνήθως, κατά την πρακτική του Transfer Learning με γλωσσικά μοντέλα, διακρίνονται δύο στάδια : Η προ-εκπαίδευση (pre-training) του μοντέλου σε εκτενές σώμα κειμένων γενικού περιεχομένου, που είναι ένα υπολογιστικά βαρύ βήμα και η τελειοποίηση (fine-tuning) του μονέλου σε συγκεκριμένα καθήκοντα, που έχει πολύ χαμηλότερο υπολογιστικό κόστος. Στην περίπτωση του BERT το πρώτο στάδιο έχει ήδη ολοκληρωθεί, αφού προ-εκπαιδευμένα μοντέλα διατίθενται δημόσια. Ωστόσο έχει διαπιστωθεί ότι η προ-εκπαίδευση πάνω σε κείμενα γενικού περιεχομένου δεν επαρκεί σε τομείς με εξειδικευμένη ορολογία, όπως βιοϊατρικά ή επιστημονικά κείμενα. Για την αντιμετώπιση αυτού του περιορισμού μπορούν να ακολουθηθούν δύο εναλλακτικές στρατηγικές : Η περαιτέρω προ-εκπαίδευση (further pre-training, FP) των δημοσίων μοντέλων σε σώματα κειμένων ειδικών τομέων, ώστε να ενσωματωθούν οι σχετικές ορολογίες, και η πλήρης προ-εκπαίδευση (from scratch, SC) του BERT σε σώματα κειμένων ειδικού τομέα. \cite{chalkidis-2020} Η μελέτη που εξετάζεται στο παρόν άρθρο, 

\subsection{Προβλεπτικά συστήματα και εφαρμογή τους στον νομικό τομέα}

\subsubsection{Εισαγωγή στα προβλεπτικά συστήματα}

\subsubsection{Εφαρμογή μεθόδων μηχανικής μάθησης}

\subsubsection{Χρήση νευρωνικών δικτύων}

\section{Κύριες αλγοριθμικές προκλήσεις}

\subsubsection{}


\subsubsection{}


\section{Εφαρμογές βασισμένες σε γράφους}

\subsubsection{}


\subsubsection{}


\section{Ηθικά και νομικά ζητήματα}

\subsection{Ύπαρξη και επιπτώσεις μεροληψίας στα δεδομένα εκπαίδευσης}

\subsubsection{Έλεγχος των δεδομένων εκπαίδευσης}


\subsubsection{Επιπτώσεις στην διαδικασία εκπαίδευσης}


\subsection{Συνέπειες στην αμεροληψία και τη δικαιοσύνη των προβλέψεων}

\subsubsection{Επιπτώσεις στην έκβαση των προβλέψεων}


\subsubsection{Διασφάλιση αξιόπιστων νομικών τεχνολογικών εργαλείων}


\section{Συμπεράσματα}


\bibliographystyle{unsrt}
\bibliography{references}
\end{document}
